Za historyczny początek makroekonomii uznaje się publikację książki Johna Maynarda Keynesa \emph{Ogólna teoria zatrudnienia, procentu i pieniądza} (\emph{General Theory of Employment, Interest, and Money}) w 1936 roku. Sukces publikacji oraz zaproponowanego nowego podejścia -- makroekonomii -- może zostać przypisany sytuacji gospodarczej panującej w latach 30. XX wieku. Keynes starał się wyjaśnić powód wystąpienia masowego bezrobocia w trakcie Wielkiego Kryzysu, który wystąpił w latach 1929-1933. Kryzys miał bardzo silny efekt na kraje zachodnie, oparte na systemie gospodarczym kapitalizmu, podczas gdy ZSRR, które opierało się o centralne planowanie oraz komunizm, doświadczyło silnego wzrostu ekonomicznego \cite{deVroey}. Dostępna teoria ekonomiczna w tym czasie sugerowała, że kryzys powinien zostać złagodzony przez deflację realnej płacy, jednak z czasem zauważono że ten efekt nie doprowadził do skrócenia czasu trwania depresji gospodarczej. Pojawiło się pytanie, czy rządowa interwencja w gospodarkę w tym czasie była wystarczająca, niestety brakowało podstawy teoretycznej do takiego działania i wielu ekonomistów bazowało te odczucia na intuicji. Książka Keynesa natomiast dostarczała teoretyczną interpretacje podstaw zajścia kryzysu oraz argumentowała za zwiększeniem wpływu rządu \cite{blanchard-macroeconomics}. W pracy zostało zaproponowane pojęcie zagregowanego popytu. Keynes argumentował, że agregowany popyt determinuje produkcję, nawet jeśli produkcja spadnie w pewnym momencie do początkowego poziomu, to efekt tego procesu jest długofalowy i powolny. Praca dodatkowo wprowadzała wiele podstawowych składowych późniejszej makroekonomii, tj. relację konsumpcji do dochodu, inwestycje autonomiczne oraz wpływ zmieniających się oczekiwań na agregowaną produkcję oraz popyt.

Korzystanie z agregowanych zmiennych opisujących wartości całej gospodarki, takie jak popyt, produkcja, inflacja lub międzynarodowy handel, stało się podstawą podejścia makroekonomicznego. Makroekonomia dostarczyła narzędzi, pozwalających na prowadzenie badań nad ogólnogospodarczymi trendami oraz zjawiskami. Szczególnie celem było wyjaśnienie zjawiska bezrobocia przymusowego -- sytuacji gdy osoby nie mogą znaleźć zatrudnienia pomimo akceptacji płacy niższej niż rynkowa \cite{deVroey} -- i podstaw teoretycznych istnienia cykli koniunkturalnych, tj. wpływowi krótkotrwałych wahań ogólnokrajowej produkcji na stan gospodarki, oraz wyznaczenie prognoz długotrwałego wzrostu gospodarczego \cite{modern_macroeconomics}. 

Nowe podejście pozwoliło także na dalszy rozwój modeli ekonomicznych. Powstały kolejne modele czerpiące inspirację z pracy Keynesa. Wśród najważniejszych z nich można wymienić model \emph{IS-LM}, zaproponowany przez Johna Hicksa w 1937 roku, model Modiglianiego z roku 1944 oraz model Kleina-Goldbergera z roku 1955, który przedstawiał gospodarkę USA \cite{deVroey}. Jednym z problemów z pracą Keynesa był brak precyzyjnego modelu, który pozwalałby na porównanie go do teorii klasycznej. Te prace łączyły teorię Keynesa z klasyczną ekonomią w celu stworzenia modeli realnej gospodarki. 

Przed przejściem do zmian zapoczątkowanych nowym podejściem ekonomicznym, pokrótce modele ekonomiczne starają się przy pomocy równań matematycznych uchwycić relację między zmiennymi makroekonomicznymi rynku. Równania modelu oddają prawa oraz znane teorie ekonomiczne, pozwalając w zwięzły sposób oddać modelowany rynek gospodarczy. Pierwsze modele ekonomiczne można znaleźć już u ekonomistów klasycznych, takich jak David Ricardo lub Karol Marks. Podejście keynesowskie wprowadziło szersze wykorzystanie agregatów do opisu zachowań komponentów rynku.

Kolejny przełom w podejściu modelowania makroekonomicznego przyszedł w latach 80. XX wieku w postaci pracy Roberta Lucasa \emph{Econometric policy evaluation: A critique}\cite{LUCAS197619}. W tej pracy została podjęta krytyka ówczesnych keynesowskich modeli makroekonomicznych, które w tamtych czasach były opracowywane pod konkretną politykę i sytuację gospodarczą. Parametry występujące w równaniach modeli były zależne od zastosowanych reguł polityki gospodarczej np. polityki monetarnej. Nawet drobna zmiana w zastosowanych regułach mogła spowodować zmianę oczekiwań oraz zachowań jednostek w gospodarce. Lucas dalej argumentuje, że wykorzystanie takich modeli nie dostarcza żadnych praktycznych informacji na temat efektów badanych reguł polityki. Pojęcia zachowań i oczekiwań jednostek są wzięte z poziomu mikroekonomicznego, stąd pojawia się pytanie w jaki sposób je zasadnie odwzorować w strukturze modeli makroekonomicznych.

Odpowiedź na to pytanie dostarczył model gospodarczy opierający się o optymalizujących agentów, opublikowany w pracy Kydlanda i Prescotta z roku 1982 \emph{Time to Build and Aggregate Fluctuations} \cite{prescott_kydland}, za którą autorom przyznano Nagrodę Nobla w dziedzinie ekonomii w 2004 roku. Ten model opierał się na połączeniu teorii cykli koniunkturalnych z teorią wzrostu gospodarczego, stąd jego późniejsza nazwa -- model realnego cyklu koniunkturalnego (\emph{Real Business Cycle Model}). Jednym z narzędzi zaprezentowanych w pracy było wprowadzenie reprezentacji dla nieuzasadnionych wahań w zmiennych ekonomicznych modelu -- nazwanych szokami. Pod tym pojęciem rozumiemy wszelkie losowe zjawiska nie objęte przez prawa i aksjomaty ekonomiczne, np. nagłe przełomy technologiczne, kryzysy lub  katastrofy klimatyczne, które prowadzą do wzrostu lub spadku produkcji. Autorzy zasugerowali opisanie szoków w postaci stochastycznych zmiennych, co pozwoliło na zgrabne ujęcie tych zewnętrznych procesów jako równań matematycznych. 

Ważnym aspektem modeli realnego cyklu koniunkturalnego jest założenie występowania w gospodarce doskonałej konkurencji, zgodnie z którą optymalizujący agenci zawsze reagują efektywnie na wahania rynkowe, dostosowując swoje zachowania do zmiennych zgodnych ze stanem równowagi ekonomicznej. Starano się także argumentować za ograniczonym wpływem czynników monetarnych na wahania rynkowe, ponieważ samo wykorzystanie szoków technologicznych pozwoliło wygenerować realistyczne prognozy, zgodne z prawdziwymi zachowaniami cyklów koniunkturalnych \cite{gali}. Rewolucja nurtu modeli realnego cyklu koniunkturalnego stała się podstawą dalszego rozwoju dziedziny modelowania ekonomicznego, która doprowadziła do powstania współczesnych modeli makroekonomicznych. 

Oparcie teorii modeli realnego cyklu koniunkturalnego o doskonałą konkurencję oraz pełną elastyczność cen niestety miało swoje ograniczenia w badaniu wpływu polityki monetarnej. Powstały prace starające się wprowadzić i zbadać wpływ polityki monetarnej w takim środowisku gospodarczym (np. model z pracy Cooleya i Hansena \emph{The Inflation Tax in a Real Business Cycle Model} \cite{cooley_hansen}), którym nie udało się dostarczyć lepszych narzędzi do badania efektów polityki gospodarczej. Odpowiedź na ten problem przyszła z nurtem tzw. Nowej Ekonomii Keynesowskiej. Modele nowokeynesowskie w przeciwieństwie do modeli realnego cyklu koniunkturalnego starały się przedstawić rynek, w którym występuje konkurencja niedoskonała. Zgodnie z tym procesem optymalizujący agenci nie zawsze muszą podejmować efektywne decyzje zgodne ze stanem równowagi modelu. Najczęściej stosowanym wariantem we współczesnych modelach ekonomicznych jest konkurencja monopolistyczna, w której w obliczu zmian firmom może np. nie opłacać się zaktualizować cen zgodnie z tą wynikającą ze stanu równowagi ekonomicznej. Z taką zmianą mogą wiązać się dodatkowe koszty lub może ona źle wyglądać w oczach zarządu lub inwestorów firmy. W związku z tym w modelach nowokeynesowskich pojawiają się mechanizmy pozwalające firmom na opóźnienie lub brak aktualizacji cen. Zastosowanie teorii Nowej Ekonomii Keynesowskiej doprowadziło do powstania współczesnych modeli DSGE (\emph{Dynamic Stochastic General Equilibrium}), łączących nieefektywności zachowań agentów z agregatami ujmującymi konsumpcję różnorodnych towarów produkowanych za równo na rynku krajowym jak i zagranicznym.

Dalszy rozwój modeli DSGE opierał się o wprowadzenie do modeli dodatkowych komponentów odpowiadających rzeczywistym elementom gospodarki. Prace Rotemberga i Woodforda \cite{doi:10.1086/654340}, Woodforda \cite{woodford_2005} oraz Galiego \cite{gali} wprowadziły do modeli zmienne oraz układy równań mechanizmów polityki monetarnej oraz różnych strategi wykorzystywanych przez banki centralne. W pracy \cite{gali} dodatkowo zaprezentowano model małej otwartej gospodarki, który opisywał zachowanie gospodarki zależnej od importu oraz eksportu, charakteryzującej się małym wpływem na wahania zmiennych makroekonomicznych na rynku globalnym. Jako przykład takiej gospodarki można wymienić Polskę, dla której zaprezentowano model w \cite{nbpKoloch}, opierający się o reprezentację z pracy Galiego. Wśród innych ważnych prac można wymienić prace Smetsa oraz Woutersa, którzy opracowali model gospodarki strefy Euro \cite{10.1162/154247603770383415} oraz gospodarki USA \cite{10.1257/aer.97.3.586}.

Równolegle z rozwojem modeli DSGE prowadzone były prace w tematyce metod rozwiązywania modeli. Rozwiązywanie modeli DSGE opiera się o stworzenie reprezentacji macierzowej dla równań oraz zmiennych ekonomicznych modelu. Praca Blancharda i Kahna \cite{10.2307/1912186} opisała metodę rozwiązywania równania macierzowego modelu oraz warunki istnienia unikalnego rozwiązania. Metoda ta zostanie szerzej opisana w rozdziale \ref{sec:blanchard_kahn_method}, jej ważnym ograniczeniem jest oparcie o warunek odwracalności jednej z macierzy układu równań modelu. Kolejne prace zaprezentowały rozwiązania pozwalające ominąć warunek odwracalności macierzy, takie jak metoda Simsa \cite{sims} oraz Kleina \cite{KLEIN20001405}. Metody te korzystając z dekompozycji QZ, pozwalały na rozwiązanie problemu nieodwracalności macierzy układu równań modelu. 

W tym miejscu warto chwile poświęcić reprezentacji równań modelu DSGE. Powyższe metody korzystają z reprezentacji układu równań modelu w postaci liniowej, w związku z tym szeroko stosowana jest aproksymacja oryginalnego układu równań do postaci liniowej. Jednym ze stosowanych w pracy narzędzi jest przedstawiona w pracy Uhliga\cite{uhlig:1995} metoda aproksymacji wokół stanu ustalonego, która w jednolity sposób aproksymuje model w postaci układu z ze zmiennymi reprezentującymi wahania od stanu ustalonego tj. niezmiennika równania macierzowego modelu, dokładniej opisanego w rozdziale \ref{sec:steady_state}. Ostatnim komponentem procesu rozwiązywania jest estymowanie wartości parametrów modelu. Parametry są zaszytymi zmiennymi modelu, w dużej mierze niezależnymi od przybranej polityki gospodarczej, opisującymi skłonności oraz relacje elementów gospodarki tj. cenowa elastyczność popytu, stopa dyskonta lub elastyczność substytucji. W związku z tym nie są one de facto zmiennymi makroekonomicznymi. Wartości tych parametrów są pozyskiwane poprzez badanie i analizę rynku, co jest w zasadzie jednym z problemów z którymi prognozowanie ekonomiczne gospodarki musi się zmierzyć. W związku z tym zostanie omówiona metoda opierająca się na wnioskowaniu bayesowskim na podstawie pracy Herbsta i Schorfheidego \cite{herbst}, pozwalająca przeprowadzić aproksymację dla wartości parametrów przy użyciu danych historycznych dla wartości parametrów oraz zmiennych gospodarczych dla poprzedzającego okresu czasu.

Obecnie modele DSGE są szeroko wykorzystywane przez banki centralne do prognozowania efektów polityki monetarnej oraz fiskalnej i rozmaitych szoków na rynku światowym. Wykorzystuje je m.in. Bank Rezerwy Federalnej w Nowym Jorku \cite{del2013frbny}, Narodowy Bank Polski \cite{nbpKoloch}, Bank Szwedzki  \cite{dsgeSweden} oraz Centralny Bank Federacji Rosyjskiej \cite{dsgeRussia}.

\section*{Cel pracy}

Podstawowym celem pracy jest zbadanie tematyki DSGE z punktu widzenia informatycznego. Jakkolwiek modele te znajdują szerokie zastosowanie w pracy współczesnych ekonomistów to brakuje omówienia algorytmiczno-matematycznej strony rozwiązywania modeli. Opierając się na bazie ekonomicznych mechanizmów i reguł zbudowany zostanie zbiór komponentów oraz zmiennych ekonomicznych wraz z wiążącymi równaniami dla różnych wariantów modelowanych gospodarek. Następnie z tych komponentów zostaną zaprezentowane pełne modele DSGE, które wykorzystane zostaną do prezentacji algorytmów estymacji oraz rozwiązywania.

Integralną częścią pracy jest repozytorium zawierające implementację opisanych algorytmów rozwiązywania modeli DSGE, która została wykonana w języku Python w wersji 3.9. Do implementacji został dołączony również zbiór przykładowych modeli wykorzystywanych w pracy.

\section*{Układ pracy}

Praca podzielona jest na trzy rozdziały. Pierwszy rozdział poświęcony jest teorii modeli DSGE. Zostaną omówione podstawy teoretyczne oraz mechanizmy modeli m.in właściwości modeli realnego cyklu koniunkturalnego oraz modeli Nowej Ekonomii Keynesowskiej. W tej części zaprezentowana jest także struktura modeli -- zmienne ekonomiczne oraz równania, reprezentujące zależności pomiędzy zmiennymi. Struktura zostanie omówiona rozpoczynając od podstawowych elementów skali mikro: gospodarstw domowych oraz firm. Następnie przedstawione zostaną rozszerzenia skali makro takie jak polityka monetarna, sektor rządowy oraz mechanizmy gospodarek opartych o import oraz eksport. Komponenty będą wzbogacone o różne warianty, pozwalające modelować różnorodne relacje oraz zachowania na rynkach ekonomicznych. 


Drugi rozdział poświęcony jest implementacji modeli. Przedstawiona zostanie postać macierzowa równań modelu oraz etapy przygotowania, estymowania oraz prognozowania modelu:
\begin{enumerate}
    \item Linearyzacja równań -- aproksymacja ułatwiająca analizę modeli wykorzystująca logarytmiczną linearyzację. Metoda opiera się na charakteryzacji stanów równowagi, poprzez aproksymację Taylora logarytmicznego odchylenia zmiennych od stanu ustalonego.
    \item Rozwiązywanie modeli -- poszukiwanie rekurencyjnej postaci dla układu równań modelu, która zachowuje oryginalny stan równowagi. Ta część skupia się na sprawdzeniu warunku istnienia unikalnego ograniczonego rozwiązania oraz rozwiązuje model w celu znalezienia postaci, która zostanie wykorzystana w następnych etapach.
    \item Estymacja parametrów -- poszukiwanie rozkładu dla parametrów modeli przy pomocy wnioskowania bayesowskiego oraz algorytmów Monte Carlo. Jednym z problemów, które modele starają się rozwiązać, jest brak dokładnych aproksymacji dla szeregu parametrów pojawiających się w modelu. Opisują one pewne ukryte skłonności na rynku, takie jak substytucja dobra, współczynnik deprecjacji kapitału lub współczynnik awersji do ryzyka. 
    \item Prognozowanie modelu -- obliczenie prognozy dla zmiennych modelu w oparciu o postać rekurencyjną oraz wy-estymowaną kalibrację parametrów. W zależności od poszukiwanego efektu możemy wygenerować sekwencję prognozowanych zmiennych lub przy pomocy algorytmu losowych ścieżek znaleźć kwantyle, estymacje punktów oraz estymacje interwałów.
\end{enumerate}
    % W tej części staram się przedstawić metodę opartą na wybieraniu losowych ścieżek dla parametru oraz metodę Blancharda-Kahna pozwalająca na prognozowanie oparte na charakteryzacji stanów równowagi dla zmiennych występujących w modelu

Ostatni rozdział pracy poświęcony jest przedstawieniu wyników zastosowanych metod. Wprowadzony zostaje model DSGE prostej gospodarki Nowej Ekonomii Keynesowskiej z wyprowadzeniem stanu ustalonego oraz linearyzacji. Następnie omówione jest repozytorium implementacji dołączone do pracy, poprzez przedstawienie struktury bazy kodu oraz zastosowanych bibliotek i ciekawych rozwiązań. Model zostaje następnie zaprezentowany w postaci pliku, podzielonego na poszczególne części opisujące zmienne, równania oraz dostępne przybliżenia parametrów oraz zmiennych modelu. Rozdział zakończony jest wynikami metod estymacji oraz prognozowania, które zostały przeprowadzone na przedstawionym liniowym modelu oraz dodatkowych modelach zaprezentowanych w tej części pracy.