\usepackage{polski}
\usepackage[utf8]{inputenc}
\usepackage{indentfirst}
\usepackage[protrusion=false]{microtype}
\usepackage{fancyhdr}
\usepackage{amssymb}
\usepackage{url,hyperref}
\hypersetup{
    colorlinks = true,
    linkcolor=.,
    citecolor=.,
    urlcolor = {blue}
}
\usepackage{multirow,multicol}
\usepackage[singlelinecheck=false,skip=2pt]{caption}
\usepackage{booktabs,float,tabularx,ltablex}
\usepackage[flushleft]{threeparttable}
\usepackage{enumitem}
\usepackage{pstricks,graphicx}
\usepackage{amsmath}
\usepackage[nottoc]{tocbibind}
\usepackage{natbib}
% \bibliographystyle{apalike}%--- Styl bibliograficzny APA 6th
\usepackage{lipsum}
\usepackage{chngcntr}
\counterwithout{table}{chapter}
\counterwithout{figure}{chapter}
\usepackage{titlesec}
\usepackage{tikz}
\usepackage{pgfplots}
\pgfplotsset{compat=1.14}
\floatstyle{plaintop}
\newfloat{wykres}{tbph}{loc}
\floatname{wykres}{Wykres}
\usepackage{cleveref}
\crefname{wykres}{wykres}{wykres}
\Crefname{wykres}{wykres}{Wykres}
\newcommand\listofcharts{\listof{wykres}{Spis wykresów}}
\usepackage{geometry}
\newgeometry{tmargin=2.5cm, bmargin=2.5cm, headheight=16pt, inner=2.5cm, outer=2.5cm} 
\usepackage{pdflscape}
\usepackage{rotating}
\usepackage{theorem}
\theoremstyle{break}
\theorembodyfont{\it}
\newtheorem{definition}{Definicja}[section]
\newtheorem{algDefinition}{Algorytm}[section]
\theorembodyfont{\rm}

\newtheorem{theorem}{Twierdzenie}
\newtheorem{defi}{Definicja}[chapter]
\usepackage{tikz}
\usepackage{listings}
\usepackage[ruled,vlined,noend]{algorithm2e}
\usepackage{minted}
\usepackage{diagbox}
\usepackage{todonotes}

% \renewcommand{\thesection}{\arabic{section}}
\SetAlgorithmName{Algorytm}{List of protocols}

%%%%%%%%%%%%%%%%%%%%%%% --- Ustawienia układu strony --- %%%%%%%%%%%%%%%%%%%%%%%

\pagestyle{fancy}
%---Druk jednostronny. Aby wyłączyć należy dodać % przed \. Aby włączyć należy usunąć % prze \.

%\fancyhead[C]{} 
%\fancyfoot[R]{\thepage}
%\fancyhead[L]{\scriptsize\leftmark}
%\fancyhead[R]{\scriptsize\rightmark}
%\cfoot[]{}
%\fancypagestyle{plain}{%
%\fancyhf{} % clear all header and footer fields
%\fancyfoot[R]{\textbf{\thepage}} % except the center
%\renewcommand{\headrulewidth}{0pt}
%\renewcommand{\footrulewidth}{0pt}}

\setcounter{secnumdepth}{4}
%---Druk dwustronny. Aby wyłączyć należy dodać % przed \. Aby włączyć należy usunąć % prze \.

\fancyhead[LE,RO]{\scriptsize\leftmark}
\fancyhead[RE,LO]{{\scriptsize \rightmark}}
\fancyfoot[CE,CO]{}
\fancyfoot[LE,RO]{\thepage}
\cfoot[]{}
\fancypagestyle{plain}{%
\fancyhf{} % clear all header and footer fields
\fancyfoot[LE,RO]{\textbf{\thepage}} % except the center
\renewcommand{\headrulewidth}{0pt}
\renewcommand{\footrulewidth}{0pt}}

%%%% Komendy dodatkowe

\newcommand{\E}{\mathbb{E}}
\newcommand{\dash}[1]{\hat{#1}}
\newcommand{\ddash}[1]{\widetilde{\widetilde{#1}}}
\newcommand{\Lagr}{\mathcal{L}}

\newcommand{\fpath}[1]{\texttt{#1}}
\newcommand{\libname}[1]{\textsc{#1}}

\newcommand{\steady}[1]{\bar{#1}_s}
\newcommand{\C}{\mathbb{C}} 

%%%%%%%%%%%%%%%%%%%%%%% --- Koniec edycji układu strony. Proszę nie edytować --- %%%%%%%%%%%%%%%%%%%%%%%
\renewcommand{\chaptermark}[1]{%
\markboth{\MakeUppercase{%
\chaptername}\ \thechapter.%
\ #1}{}}
\renewcommand{\sectionmark}[1]{\markright{\thesection.\ #1}}

\colorlet{punct}{red!60!black}
\definecolor{background}{HTML}{EEEEEE}
\definecolor{delim}{RGB}{20,105,176}
\colorlet{numb}{magenta!60!black}

\lstdefinelanguage{json}{
    basicstyle=\normalfont\ttfamily,
    numbers=left,
    numberstyle=\scriptsize,
    stepnumber=1,
    numbersep=8pt,
    showstringspaces=false,
    breaklines=true,
    frame=lines,
    backgroundcolor=\color{background},
    literate=
     *{0}{{{\color{numb}0}}}{1}
      {1}{{{\color{numb}1}}}{1}
      {2}{{{\color{numb}2}}}{1}
      {3}{{{\color{numb}3}}}{1}
      {4}{{{\color{numb}4}}}{1}
      {5}{{{\color{numb}5}}}{1}
      {6}{{{\color{numb}6}}}{1}
      {7}{{{\color{numb}7}}}{1}
      {8}{{{\color{numb}8}}}{1}
      {9}{{{\color{numb}9}}}{1}
      {:}{{{\color{punct}{:}}}}{1}
      {,}{{{\color{punct}{,}}}}{1}
      {\{}{{{\color{delim}{\{}}}}{1}
      {\}}{{{\color{delim}{\}}}}}{1}
      {[}{{{\color{delim}{[}}}}{1}
      {]}{{{\color{delim}{]}}}}{1},
}

\usetikzlibrary{shapes, arrows, arrows.meta, calc,fit, backgrounds, shapes.multipart, positioning}
\tikzset{box/.style={draw, rectangle, rounded corners, thick, node 
distance=7em, 
text width=6em, text centered, minimum height=3.5em}}
%\tikzset{line/.style={draw, thick, -{Latex[length=2mm,width=1mm]}}}
\tikzset{every node/.style={font=\scriptsize}}

\lstdefinelanguage{args}{
sensitive=false,
alsoletter={.},
moredelim=[s][\color{red}]{<}{>},
moredelim=[s][\color{blue}]{[}{]},
moredelim=[is][\color{orange}]{:}{:},
keywords=[10]{...},
keywordstyle=[10]{\color{magenta}},
}

\lstnewenvironment{arguments}
{\lstset{language=args}}
{}

\makeatletter
\renewcommand{\fnum@figure}{Rys. \thefigure}
\makeatother