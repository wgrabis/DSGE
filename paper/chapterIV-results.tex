
Ten rozdział poświęcony został wynikom implementacji oraz symulacji modeli DSGE. W części \ref{sec:results:model} został wyznaczony stan ustalony oraz obczliona aproksymacja liniowa dla modelu gospodarki z rozdziału \ref{sec:sample_models}. W sekcji \ref{sec:full_linear_model_sample} przedstawiony został pełny liniowy model, wraz z kalibracją parametrów niezbędną do przeprowadzenia prognozowania metodą Blancharda-Kahna. Następna część \ref{sec:implementation_result} poświęcona jest przedstawieniu solucji implementującej metody z rozdziału \ref{chapter:implementation}, rozpoczynając od struktury folderów repozytorium oraz krótkiego opisu wykorzystanych bibliotek, kończąc opisem pliku modelu na podstawie modelu zaprezentowanego na początku rozdziału.

Zakończenie rozdziału zostało poświęcone opracowaniu wyników prognozowania dla wybranych modeli oraz różnych szoków impulsowych. Jako pierwszy przedstawiono model z rozdziału \ref{sec:results:model} dla dwóch szoków: produktywności oraz polityki monetarnej. Kolejnym jest model RBC, który został przedstawiony w sekcji \ref{sec:rbc_model_sample}, dla predykcji zachowania szoku produktywności. Na koniec zaprezentowano model gospodarki Irlandii z 2004 roku oparty o pracę \cite{NBERw10309} z estymacją bayesowską parametrów dla sekwencji kilkudziesięciu lat danych gospodarczych.

\section{Linearyzacja modelu NEK}
\label{sec:results:model}

W celu prezentacji wyników pracy zostanie zastosowany model przedstawiony w rozdziale \ref{sec:nek_model_sample}. Poniżej postać modelu w postaci nieliniowej:
\begin{itemize}
    \item Układ równań problemu optymalizacji użyteczności gospodarstw domowych:
        \begin{gather}
            W_t = L_t^{\phi} C_t^{\sigma}, \\
            \frac{1}{R^Q_t} = \beta \E_t \left( \left(\frac{C_t}{C_{t+1}}\right)^\sigma \frac{1}{\pi_{t+1}}\right), \\
            \frac{1}{\beta}\E_t\left(\frac{C_{t+1}}{C_t}\right)^\sigma = (1-\delta) + \E_t R^K_{t+1}, \\
            K_{t+1} = (1-\delta)K_t + I_t.
        \end{gather}
    \item Rozwiązanie komponentu firm:
        \begin{align}
            Y_{j,t} &= A_t K_{j,t}^\alpha L_{j,t}^{1-\alpha}, \\
            L_{j,t} &= (1-\alpha) MC_{j,t} \frac{Y_{j,t}}{W_t},\\
            K_{j,t} &= \alpha MC_{j,t} \frac{Y_{j,t}}{R^K_t},\\
            \log{A_t} &= (1-\rho_A)\log{\steady{A}} + \rho_A \log{A_{t-1}} + \epsilon_{A,t}.
        \end{align}
    \item Układ równań opisujący poziom cen:
        \begin{align}
            \label{eq:price_level_firm_sample}
            0 &= \E_t \sum_{i=0}^{\infty}\left( \beta \theta \right)^i Y_{j,t+i} \left( \frac{P^*_{j,t}}{P_{t+i}} - \frac{\psi}{\psi - 1} MC_{j,t+i}\right), \\
            \label{eq:price_level_firm_all_sample}
            P_t^{1-\psi} &= \theta P_{t-1}^{1-\psi} + (1-\theta) P_{t}^{*\;1-\psi}, \\ 
            \Pi_t &= \frac{P_t}{P_{t-1}}.
        \end{align}
    \item Równanie czyszczenia rynku:
        \begin{align}
            Y_t &= C_t + I_t.
        \end{align}
    \item Polityka monetarna:
        \begin{gather}
            R^Q_t = R_t^{Q^*(1-\rho_Q)}R_{t-1}^{Q\rho_Q}, \\
            R^{Q^*}_t = \steady{R^Q} \pi^* \left( \frac{\pi_t}{\pi^*}\right)^{\psi_1} \left(\frac{y_t}{\steady{y}}\right)^{\psi_2}.
    \end{gather}
\end{itemize}
W kolejnych sekcjach zostanie przeprowadzony proces aproksymacji liniowej, rozpoczynając od obliczenia stanu ustalonego. W ostatniej sekcji \ref{sec:full_linear_model_sample} zaprezentowany został końcowy model. Warto w tym miejscu nadmienić, że zmienne indeksowane firmą $j$ zostaną sprowadzone do postaci bez indeksu $j$. Wynika to z faktu, że problem jest symetryczny dla każdej z firm i wystarczy przedstawić rozwiązanie dla jednej reprezentatywnej zmiennej.

\subsection{Stan ustalony}
\label{sec:results:steady_state_model}

Pierwszym krokiem niezbędnym do ustalenia aproksymacji liniowej modelu jest znalezienie stanu ustalonego. Sprowadzamy zmienne do postaci bez sztywności, tj. podstawiamy $y^*_{t} = y_{t}$ a następnie zastępujemy $\steady{y} = \E_t y_{t+1} = y_t = y_{t-1}$, otrzymując układ równań:

\begin{gather}
    \label{eq:mc_consumption_solve}
    \steady{W} = \steady{L}^{\phi} \steady{C}^{\sigma}, \\
    \label{eq:steady_monetary_solve}
    \frac{1}{\steady{R^Q}} = \beta  \left(\frac{\steady{C}}{\steady{C}}\right)^\sigma \frac{1}{\steady{\pi}}, \\
    \label{eq:steady_r_solve}
    \frac{1}{\beta}\left(\frac{\steady{C}}{\steady{C}}\right)^\sigma = (1-\delta) + \steady{R^K}, \\
    \label{eq:capital_movement_steady_solve}
    \steady{K} = (1-\delta)\steady{K} + \steady{I},\\
    \label{eq:mc_production_solve}
    \steady{Y} = \steady{A} \steady{K}^\alpha \steady{L}^{1-\alpha}, \\
    \label{eq:mc_worktime_solve}
    \steady{L} = (1-\alpha) \steady{MC} \frac{\steady{Y}}{\steady{W}},\\
    \label{eq:mc_invest_solve}
    \steady{K} = \alpha \steady{MC} \frac{\steady{Y}}{\steady{R}^K},\\
    \log{\steady{A}} = (1-\rho_A)\log{\steady{A}} + \rho_A \log{\steady{A}},\\
    \label{eq:mc_steady_solve}
    0 = \E_t \sum_{i=0}^{\infty}\left( \beta \theta \right)^i \steady{Y} \left( \frac{\steady{P}}{\steady{P}} - \frac{\psi}{\psi - 1} \steady{MC} \right), \\
    \steady{P}^{1-\psi} = \theta \steady{P}^{1-\psi} + (1-\theta) \steady{P}^{1-\psi}, \\
    \steady{\Pi} = \frac{\steady{P}}{\steady{P}},\\
    \steady{Y} = \steady{C} + \steady{I}, \\
    \steady{R^Q} = \steady{R}^{Q^*(1-\rho_Q)}\steady{R}^{Q\rho_Q}, \\
    \steady{R}^{Q} = \steady{R}^Q \pi^* \left( \frac{\steady{\pi}}{\pi^*}\right)^{\psi_1} \left(\frac{\steady{y}}{\steady{y}}\right)^{\psi_2}.
\end{gather}

W pierwszym kroku wyeliminujemy trywialne równania oraz znormalizujemy zmienną produkcji. Możemy zauważyć że w stanie ustalonym cena nie zmienia się w czasie, stąd otrzymujemy:
\begin{equation}
    \steady{\Pi} = 1.
\end{equation}
Dodatkowo zgodnie ze wspomnianą normalizacją zakładamy:
\begin{align}
    \steady{A} &= 1.
\end{align}
Z tych zależności możemy następnie wyznaczyć stan ustalony dla pozostałych zmiennych. Korzystając z równania \eqref{eq:steady_r_solve} po uproszczeniu dostajemy:
\begin{equation}
    \steady{R}^K = \frac{1}{\beta} - (1 - \delta).
\end{equation}
Stopa procentowa w stanie ustalonym, korzystając z równania \eqref{eq:steady_monetary_solve}, wynosi:
\begin{equation}
    \steady{R}^Q = \frac{1}{\beta}.
\end{equation}
Następnie wychodząc z równania \eqref{eq:mc_steady_solve} otrzymujemy wartość stanu ustalonego zmiennej kosztu krańcowego $\steady{MC}$:
\begin{equation}
    \steady{MC} = \frac{\psi - 1}{\psi}.
\end{equation}
Korzystając kolejno z wartości $\steady{MC}$ oraz równań dla układu pracy \eqref{eq:mc_worktime_solve} i \eqref{eq:mc_invest_solve}, otrzymujemy:
\begin{equation}
    \steady{L}^{1-\alpha}\steady{K}^\alpha = (1-\alpha)^{(1-\alpha)} \alpha^\alpha \steady{MC} \steady{Y} \frac{1}{\steady{W}^{(1-\alpha)} \steady{R}^{K\alpha}}.
\end{equation}
Podstawiając równanie \eqref{eq:mc_production_solve}:
\begin{equation}
    \frac{\steady{Y}}{\steady{A}} = (1-\alpha)^{(1-\alpha)} \alpha^\alpha \steady{MC} \steady{Y} \frac{1}{\steady{W}^{(1-\alpha)} \steady{R}^{K\alpha}}.
\end{equation}
Po reorganizacji dostajemy stan ustalony dla realnej płacy:
\begin{equation}
    \steady{W} = (1-\alpha)\left(\steady{A} \steady{MC} \left(\frac{\alpha}{\steady{R}^{K}}\right)^\alpha\right)^{\frac{1}{1-\alpha}}.
\end{equation}
W celu znalezienia wartości stanu ustalonego dla produkcji $Y$, kapitału $K$, inwestycji $I$ oraz konsumpcji $C$, najpierw skorzystamy z równania opisującego prawo ruchu kapitału \eqref{eq:capital_movement_steady_solve}. Opiszmy stan ustalony kapitału przez inwestycje:
\begin{equation}
    \steady{K} = \frac{1}{\delta} \steady{I}.
\end{equation}
Następnie z równania \eqref{eq:mc_invest_solve} mamy:
\begin{equation}
    \steady{I} = \delta \alpha \steady{MC} \frac{\steady{Y}}{\steady{R}^K},
\end{equation}
a dla czasu pracy z równania \eqref{eq:mc_worktime_solve}:
\begin{equation}
    \steady{L} = (1-\alpha)\steady{MC}\frac{\steady{Y}}{\steady{W}}.
\end{equation}
Kolejnym krokiem jest podstawienie równania \eqref{eq:mc_worktime_solve} do równania \eqref{eq:mc_consumption_solve}:
\begin{equation}
    \steady{C} = \left( \steady{W} \left( \frac{\steady{W}}{\steady{Y}(1-\alpha)\steady{MC}}\right)^\phi\right)^{\frac{1}{\sigma}}.
\end{equation}
Pozostaje w powyższych zależność od stanu ustalonego produkcji $Y$, stąd skorzystamy z równania czyszczenia rynku \eqref{eq:mc_production_solve}:
\begin{align}
    \steady{Y} &= \steady{C} + \steady{I} \\ 
    &= \left( \steady{W} \left( \frac{\steady{W}}{\steady{Y}(1-\alpha)\steady{MC}}\right)^\phi\right)^{\frac{1}{\sigma}} + \delta \alpha \steady{MC} \frac{\steady{Y}}{\steady{R}^K},
\end{align}
z którego po uporządkowaniu otrzymujemy:
\begin{equation}
    \steady{Y} = \left( \steady{W} \left( \frac{\steady{W}}{(1-\alpha)\steady{MC}}\right)^\phi\right)^{\frac{1}{\sigma + \phi}} \frac{\steady{R}^K}{\steady{R}^K - \delta \alpha \steady{MC}}^{\frac{\sigma}{\sigma + \phi}}.
\end{equation}
Ostatecznie stan ustalony dla zmiennych modelu prezentuje się następująco:
\begin{align}
    \steady{A} &= 1, \\
    \steady{\Pi} &= 1, \\
    \steady{R}^Q &= \frac{1}{\beta}, \\
    \steady{R}^K &= \frac{1}{\beta} - (1 - \delta), \\
    \label{eq:mc_steady_solution}
    \steady{MC} &= \frac{\psi - 1}{\psi}, \\
    \steady{W} &= (1-\alpha)\left(\steady{A} \steady{MC} \left(\frac{\alpha}{\steady{R}^{K}}\right)^\alpha\right)^{\frac{1}{1-\alpha}}, \\
    \steady{Y} &= \left( \steady{W} \left( \frac{\steady{W}}{(1-\alpha)\steady{MC}}\right)^\phi\right)^{\frac{1}{\sigma + \phi}} \frac{\steady{R}^K}{\steady{R}^K - \delta \alpha \steady{MC}}^{\frac{\sigma}{\sigma + \phi}}, \\
    \steady{L} &= (1-\alpha)\steady{MC}\frac{\steady{Y}}{\steady{W}}, \\
    \steady{C} &= \left( \steady{W} \left( \frac{\steady{W}}{\steady{Y}(1-\alpha)\steady{MC}}\right)^\phi\right)^{\frac{1}{\sigma}}, \\
    \steady{I} &= \delta \alpha \steady{MC} \frac{\steady{Y}}{\steady{R}^K}, \\
    \steady{K} &= \steady{I}\frac{1}{\delta}.
\end{align}

\subsection{Linearyzacja równań}

Korzystając z wartości stanu ustalonego możemy przeprowadzić aproksymację liniową układu równań naszego modelu. Linearyzacja równania podaży pracy przebiega następująco:
\begin{gather}
    C_t^\sigma L_s^\phi = W_t, \\
    \steady{C} e^{\sigma c_t} \steady{L} e^{\phi l_t} = \steady{W} e^{w_t}.
\end{gather}
Równanie dla stanu ustalonego $\steady{C}\steady{L} = \steady{W}$ pozwala uprościć równość do postaci:
\begin{equation}
    e^{\sigma c_t + \phi l_t} = e^{{w_t}}.
\end{equation}
Ostatnim krokiem jest zastosowanie reguły aproksymacji $e^{\hat{x}_t + \alpha \hat{y}_t} \approx 1 + \hat{x}_t + \alpha \hat{y}_t$:
\begin{gather}
    1 + \sigma c_t + \phi l_t = 1 + w_t, \\
    \sigma c_t + \phi l_t = w_t.
\end{gather}
Sprowadzenie równania Eulera wiążącego obligacje przebiega następująco:
\begin{gather}
    \frac{1}{R^Q_t} = \beta \E_t \left( \left(\frac{C_t}{C_{t+1}}\right)^\sigma \frac{1}{\pi_{t+1}}\right), \\
    \frac{1}{\steady{R}^Qe^{r^Q_t}} = \beta \E_t \left( \left(\frac{\steady{C}e^{c_t}}{\steady{C}e^{c_{t+1}}}\right)^\sigma \frac{1}{\steady{\pi}e^{\pi_{t+1}}}\right).
\end{gather}
Ponownie korzystamy z definicji stanu ustalonego $\steady{R}^Q = \frac{1}{\beta}$ oraz uporządkowujemy:
\begin{equation}
    e^{-r^Q_t} = \E_t \left( e^{\sigma c_t - \sigma c_{t+1}-\pi_{t+1}}\right).
\end{equation}
Korzystamy z przybliżenia przy pomocy szeregu Taylora:
\begin{equation}
    1 - r^Q_t = \E_t \left( 1 + \sigma(c_t - c_{t+1}) - \pi_{t+1}\right).
\end{equation}
Po reorganizacji postać liniowa prezentuje się następująco:
\begin{equation}
    \sigma(\E_t c_{t+1} - c_t) = r^Q_t - \E_t \pi_{t+1}.
\end{equation}
Aproksymacja liniowa równania Eulera dla kapitału przebiega w analogiczny sposób:
\begin{gather}
    \frac{1}{\beta} \left( \frac{\E_t C_{t+1}}{C_t}\right)^\sigma = (1-\delta) + \E_t R^K_{t+1}, \\
    \frac{1}{\beta} \frac{\steady{C}^\sigma \E_t e^{\sigma c_{t+1}}}{\steady{C}^\sigma e^{\sigma c_t}} 
    = 
    (1-\delta) + \E_t \left( \steady{R}^K e^{r^K_{t+1}}\right), \\
    \frac{1}{\beta} e^{\E_t\sigma c_{t+1} - \sigma c_t}
    = 
    (1-\delta) + \steady{R}^K e^{\E_t r^K_{t+1}}.
\end{gather}
Ponownie korzystamy z reguły aproksymacji:
\begin{gather}
    \frac{1}{\beta}\left(1 + \sigma(\E_t c_{t+1} - c_t)\right) =  (1 - \delta) + \steady{R}^K\left( 1 + \E_t r^K_{t+1}\right), \\
    \frac{1}{\beta}\ + \frac{\sigma}{\beta} \left(\E_t c_{t+1} - c_t\right) = (1 - \delta) + \steady{R}^K + \steady{R}^K \E_t r^K_{t+1}.
\end{gather}
Ostatnim etapem jest uproszczenie równania, wiedząc że w stanie ustalonym zachodzi:
\begin{equation}
    \frac{1}{\beta} = (1-\delta) + \steady{R}^K,
\end{equation}
skąd dostajemy postać aproksymowaną:
\begin{equation}
    \frac{\sigma}{\beta}(\E_t c_{t+1} - c_t) = \steady{R}^K \E_t r^K_{t+1}.
\end{equation}
Ostatnim równaniem układu gospodarstw domowych jest reguła ruchu (deprecjacji) kapitału. Linearyzacja przebiega następująco:
\begin{gather}
     K_{t+1} = (1-\delta)K_t + I_t \\
     \steady{K} e^{k_{t+1}} = (1-\delta)\steady{K} e^{k_t} + \steady{I} e^{i_t}.
\end{gather}
Korzystamy z reguły aproksymacji $e^{\hat{x}_t} \approx 1 + \hat{x}_t$:
\begin{equation}
    \steady{K} (1 + k_{t+1}) = (1-\delta)\steady{K} (1 + k_t) + \steady{I} (i_t + 1).
\end{equation}
Następnie dla stanu ustalonego mamy zachowane $\delta\steady{K} = \steady{I}$:
\begin{gather}
    \steady{K} (1 + k_{t+1}) = (1-\delta)\steady{K} (1 + k_t) + \delta\steady{K} (i_t + 1), \\
    1 + k_{t+1} = (1-\delta)(1 + k_t) + \delta(i_t + 1), \\
    1 + k_{t+1} = (1-\delta)k_t + \delta i_t + 1 - \delta + \delta.
\end{gather}
Końcowa postać liniowa:
\begin{equation}
    k_{t+1} = (1-\delta)k_t + \delta i_t.
\end{equation}

Kolejnym komponentem modelu jest układ równań problemu firm. Aproksymacja liniowa funkcji produkcji wokół stanu ustalonego prezentuje się następująco:
\begin{gather}
    Y_t = A_t K_t^\alpha L_t^{1-\alpha}, \\
    \steady{Y} e^{y_t} = \steady{A} e^{a_t} \steady{K}^\alpha e^{\alpha k_t} \steady{L}^{1-\alpha} e^{(1-\alpha) l_t}, \\
    \steady{Y} e^{y_t} = \steady{A} \steady{K}^\alpha \steady{L}^{1-\alpha} e^{a_t + \alpha k_t + (1-\alpha)l_t}, \\ 
    \steady{Y} (1 + y_t) = \steady{A} \steady{K}^\alpha \steady{L}^{1-\alpha} \left( 1+ a_t + \alpha k_t + (1-\alpha)l_t \right).
\end{gather}
Analogicznie do poprzednich przykładów, skorzystamy z równowagi w stanie ustalonym $\steady{Y} = \steady{A} \steady{K}^\alpha \steady{L}^{1-\alpha}$ dostając liniowe równanie:
\begin{equation}
    y_t = a_t + \alpha k_t + (1-\alpha) l_t.
\end{equation}
Równanie zapotrzebowania na pracę, stosując dla stanu ustalonego $\steady{L} = (1-\alpha) \steady{MC} \frac{\steady{Y}}{\steady{W}}$, jest aproksymowane przez przekształcenia:
\begin{gather}
    L_t = (1 - \alpha)MC_t \frac{Y_t}{W_t}, \\
    \steady{L} e^{l_t} = (1-\alpha) \steady{MC} e^{mc_t} \frac{\steady{Y} e^{y_t}}{\steady{W} e^{w_t}}, \\
    \steady{L} e^{l_t} = (1-\alpha) \steady{MC}  \frac{\steady{Y}}{\steady{W}} e^{mc + y_t - w_t}, \\
    1 + l_t = mc_t + y_t - w_t + 1, \\ 
    l_t = mc_t + y_t - w_t.
\end{gather}
Kolejnym składnikiem komponentu jest równanie zapotrzebowania na kapitał, analogicznie aproksymowane z wykorzystaniem $\steady{K} = \steady{MC} \frac{\steady{Y}}{\steady{R}}$:
\begin{gather}
    K_t = \alpha MC_t \frac{Y_t}{R^K_t}, \\
    \steady{K} e^{k_t} = \alpha \steady{MC} e^{mc_t} \frac{\steady{Y} e^{y_t}}{\steady{R}^K e^{r^K_t}}, \\
    \steady{K} e^{k_t} = (1-\alpha) \steady{MC}  \frac{\steady{Y}}{\steady{R}^K} e^{mc + y_t - r^K_t}, \\
    1 + k_t = 1+ mc_t + y_t - r^K_t, \\
    k_t = mc_t + y_t - r^K_t.
\end{gather}
Równanie opisujące produktywność przy założeniu $\steady{A} = 1$:
\begin{gather}
    \log A_t = \log \steady{A} + \rho_A \log A_{t-1} + \epsilon_{A,t}, \\
    \log e^{a_t} = \rho_A \log e^{a_{t-1}} + \epsilon_t, \\
    a_t = \rho_A a_{t-1} + \epsilon_{A,t}.
\end{gather}

Komponent odpowiedzialny za poziom cen oraz inflację składa się z trzech równań, które zostaną uproszczone i połączone w jedno liniowe równanie. Rozpoczniemy od znalezienia równania dla poziomu cen ustalanego przez aktualizujące firmy \eqref{eq:price_level_firm_sample}:
\begin{equation}
    0 = \E_t \sum_{i=0}^{\infty}\left( \beta \theta \right)^i Y_{t+i} \left( \frac{P^*_{t}}{P_{t+i}} - \frac{\psi}{\psi - 1} MC_{t+i}\right).
\end{equation}
Wprowadzając zmienne odchylenia od stanu ustalonego:
\begin{equation}
    0 = \E_t \sum_{i=0}^{\infty}\left( \beta \theta \right)^i \steady{Y} e^{y_{t+i}} \left( \frac{\steady{P}e^{p^*_{t}}}{\steady{P}e^{p_{t+i}}} - \frac{\psi}{\psi - 1} \steady{MC} e^{mc_{t+i}}\right).
\end{equation}
Porządkując oraz korzystając ze stanu ustalonego $1 = \frac{\psi}{\psi - 1} \steady{MC}$ dostajemy:
\begin{equation}
    0 = \E_t \sum_{i=0}^{\infty}\left( \beta \theta \right)^i \left( e^{y_{t+i} + p^*_{t} - p_{t+i}} - e^{y_{t+i} + mc_{t+i}}\right).
\end{equation}
Następnie stosujemy aproksymację szeregiem Taylora:
\begin{gather}
    0 = \E_t \sum_{i=0}^{\infty}\left( \beta \theta \right)^i \left( 1 + y_{t+i} + p^*_{t} - p_{t+i} - (1 + y_{t+i} + mc_{t+i})\right), \\
    0 = p^*_{t} \sum_{i=0}^{\infty}\left( \beta \theta \right)^i - \E_t \sum_{i=0}^{\infty}\left( \beta \theta \right)^i \left(  mc_{t+i} + p_{t+i} \right), \\
    0 =  p^*_{t} \frac{1}{1-\beta\theta} - \E_t \sum_{i=0}^{\infty}\left( \beta \theta \right)^i \left(  mc_{t+i} + p_{t+i} \right), \\
    p^*_t = (1-\beta\theta)\E_t \sum_{i=0}^{\infty}\left( \beta \theta \right)^i \left(  mc_{t+i} + p_{t+i} \right)
    \label{eq:price_level_approx_def}.
\end{gather}
Przed zastosowaniem powyższego potrzebujemy uprościć równanie reguły poziomu cen z wyceny Calvo \eqref{eq:price_level_firm_all_sample}:
\begin{gather}
    P_t^{1-\psi} = \theta P_{t-1}^{1-\psi} + (1-\theta) P_{t}^{*\;1-\psi}, \\
    \steady{P}e^{p_t(1-\psi)} = \theta \steady{P}^{1-\psi} e^{p_{t-1}(1-\psi)} + (1-\theta) \steady{P}^{1-\psi} e^{p_t^*(1-\psi)}.
\end{gather}
Uproszczenie $\steady{P}$ oraz zastosowanie aproksymacji szeregiem Taylora:
\begin{gather}
    \left(1 + p_t + (1- \psi)\right) = \theta\left(1 + p_{t-1} + (1-\psi)\right) + (1-\theta)\left(1 + p_t^* + (1-\psi)\right),\\
    p_t = \theta p_{t-1} + (1 - \theta)p_t^*.
\end{gather}
W tej postaci możemy podstawić wyrażenie dla $p_t^*$ z \eqref{eq:price_level_approx_def}:
\begin{equation}
    p_t = \theta p_{t-1} + (1 - \theta) \left( 1 - \beta \theta \right) \E_t \sum_{i=0}^{\infty} \left( \beta \theta \right)^i \left(  mc_{t+i} + p_{t+i} \right).
\end{equation}
Następnie zastosujemy metodę quasi-różniczkowania (\emph{quasi-differencing} \cite{costaBook}) w celu wyeliminowania sumy nieskończonej przewidywanych kosztów krańcowych z równania. W tym celu wprowadzimy operator opóźnienia $L^{-1}$, dla którego zachodzi $L^{-1} X_t = X_{t+1}$. Mnożąc obustronnie przez $(1-\beta\theta L^{-1})$ otrzymujemy:
\begin{gather}
    \nonumber p_t - \beta\theta p_{t+1} = \theta p_{t-1} - \beta \theta^2 p_t + (1 - \theta)(1 - \beta \theta) \E_t \sum_{i=0}^{\infty} \left( \beta \theta \right)^i \left(  mc_{t+i} + p_{t+i} \right)\\ - (1 - \theta)(1 - \beta \theta) \E_t \sum_{i=1}^{\infty} \left( \beta \theta \right)^i \left(  mc_{t+i} + p_{t+i} \right), \\
    p_t - \beta\theta p_{t+1} = \theta p_{t-1} - \beta \theta^2 p_t + (1 - \theta)(1 - \beta \theta) \left(mc_t + p_t \right).
\end{gather}
Kolejno porządkujemy:
\begin{gather}
    \nonumber p_t - \beta\theta p_{t+1} = \theta p_{t-1} - \beta \theta^2 p_t + (1 - \theta)(1 - \beta \theta) mc_t \\ + (1 - \theta)(1 - \beta \theta) p_t, \\
    \theta(p_t - p_{t-1}) = \beta \theta (p_{t+1} - p_t) + (1-\theta)(1-\beta \theta)mc_t, \\
    \label{eq:phillips_equation}
    p_t - p_{t-1} = \beta (p_{t+1} - p_t) + \frac{(1-\theta)(1-\beta \theta)}{\theta}mc_t.
\end{gather}
Powyższe równanie nazywane jest równaniem Phillipsa \cite{costaBook}. Ostatnim elementem komponentu poziomu cen jest równanie definiujące inflację, korzystamy z $\steady{\Pi} = 1$:
\begin{gather}
    \Pi_t = \frac{P_t}{P_{t-1}}, \\
    \steady{\Pi} e^{\pi_t} = \frac{\steady{P}e^{p_t}}{\steady{P}e^{p_{t-1}}} = e^{p_t - p_{t-1}},\\ 
    1 + \pi_t = 1 + p_t - p_{t-1}, \\
    \pi_t = p_t - p_{t-1}.
\end{gather}
Uposażeni w postać liniową inflacji możemy uprościć zapis równania Philipsa \eqref{eq:phillips_equation}:
\begin{equation}
    \pi_t = \beta \pi_{t+1} + \frac{(1-\theta)(1-\beta \theta)}{\theta}mc_t.
\end{equation}

Pozostałą częścią modelu są równania polityki monetarnej oraz czyszczenia rynku. Aproksymacja tego ostatniego przebiega następująco:
\begin{gather}
    Y_t = C_t + I_t \\
    \steady{Y}e^{y_t} = \steady{C}e^{c_t} + \steady{I}e^{i_t}, \\
    \steady{Y}(1 + y_t) = \steady{C}(1 + c_t) + \steady{I}(1 + i_t), \\
    \steady{Y}y_t = \steady{C} c_t + \steady{I} i_t,
\end{gather}
gdzie w ostatnim kroku skorzystaliśmy z równowagi stanu ustalonego $\steady{Y} = \steady{C} + \steady{I}$. 

Reguła opóźnienia realnej stopy procentowej:
\begin{gather}
    R^Q_t = R_t^{Q^*1-\rho_Q} R_{t-1}^{Q\rho_Q}, \\
    \steady{R}^Q e^{r^q_t} = \steady{R}^Q e^{r^{q^*}_t(1-\rho_Q)} e^{r^q_{t-1}\rho_Q}, \\
    1 + r^q_t = 1 + (1-\rho_Q)r^{q^*}_t + r^q_{t-1} \rho_Q , \\
    r^q_t = (1-\rho_Q)r^{q^*}_t + \rho_Q r^q_{t-1}.
\end{gather}
Ostatnim równaniem modelu jest równanie nominalnej stopy procentowej ustalanej przez bank centralny:
\begin{gather}
    R^{Q^*}_t = \steady{R}^Q \pi^* \left( \frac{\pi_t}{\pi^*}\right)^{\psi_1} \left(\frac{y_t}{\steady{y}}\right)^{\psi_2}, \\
    \steady{R}^Q e^{r_t^{Q^*}} = \steady{R}^Q \pi^* \left( \frac{\steady{\pi} e^{\pi_t}}{\pi^*}\right)^{\psi_1} \left(\frac{\steady{y} e^{y^*_t}}{\steady{y}}\right)^{\psi_2}, \\
    e^{r^{Q^*}_t} = \pi^{*1-\psi_1} \left(\steady{\pi} e^\pi_t\right)^{\psi_1} y_t^{*\psi_2}, \\
    1 + r^{Q^*}_t = \pi^{*1-\psi_1} ( 1 + \psi_1 \pi_t + \psi_2 y^*_t).
\end{gather}

\subsection{Pełna postać modelu}
\label{sec:full_linear_model_sample}

Ostatecznie nasz model składa się z:
\begin{enumerate}
    \item Definicji wartości stanu ustalonego:
        \begin{align}
                \steady{A} &= 1, \\
                \steady{\Pi} &= 1, \\
                \steady{R}^Q &= \frac{1}{\beta}, \\
                \steady{R}^K &= \frac{1}{\beta} - (1 - \delta), \\
                \steady{MC} &= \frac{\psi - 1}{\psi}, \\
                \steady{W} &= (1-\alpha)\left(\steady{A} \steady{MC} \left(\frac{\alpha}{\steady{R}^{K}}\right)^\alpha\right)^{\frac{1}{1-\alpha}}, \\
                \steady{Y} &= \left( \steady{W} \left( \frac{\steady{W}}{(1-\alpha)\steady{MC}}\right)^\phi\right)^{\frac{1}{\sigma + \phi}} \frac{\steady{R}}{\steady{R} - \delta \alpha \steady{MC}}^{\frac{\sigma}{\sigma + \phi}}, \\
                \steady{L} &= (1-\alpha)\steady{MC}\frac{\steady{Y}}{\steady{W}}, \\
                \steady{C} &= \left( \steady{W} \left( \frac{\steady{W}}{\steady{Y}(1-\alpha)\steady{MC}}\right)^\phi\right)^{\frac{1}{\sigma}}, \\
                \steady{I} &= \delta \alpha \steady{MC} \frac{\steady{Y}}{\steady{R}}, \\
                \steady{K} &= \steady{I}\frac{1}{\delta}.
        \end{align}
    \item Układu liniowych równań modelu:
    \begin{itemize}
        \item Część gospodarstw domowych:
        \begin{gather}
            \sigma c_t + \phi l_t = w_t, \\
            \sigma(\E_t c_{t+1} - c_t) = r^Q_t - \E_t \pi_{t+1}, \\
            \frac{\sigma}{\beta}(\E_t c_{t+1} - c_t) = \steady{R}^K \E_t r^K_{t+1}, \\
            k_{t+1} = (1-\delta)k_t + \delta i_t.
        \end{gather} 
        \item Część firm:
        \begin{gather}
            y_t = a_t + \alpha k_t + (1-\alpha) l_t,\\
            k_t = mc_t + y_t - r^K_t,\\
            l_t = mc_t + y_t - w_t,\\
            a_t = \rho_A a_{t-1} + \epsilon_{A,t}.
        \end{gather}
        \item Poziom cen:
        \begin{gather}
            \pi_t = \beta \pi_{t+1} + \frac{(1-\theta)(1-\beta \theta)}{\theta}mc_t.
        \end{gather}
        \item Równanie czyszczenia rynku:
        \begin{equation}
            \steady{Y}y_t = \steady{C} c_t + \steady{I} i_t.
        \end{equation}
        \item Polityka monetarna:
        \begin{gather}
            r^q_t = (1-\rho_Q)r^{q^*}_t + \rho_Q r^q_{t-1}, \label{eq:monetary_nek_full_final}\\
            1 + r^{Q^*}_t = \pi^{*1-\psi_1} ( 1 + \psi_1 \pi_t + \psi_2 y^*_t).
        \end{gather}
    \end{itemize}
    \item Kalibracji parametrów, wziętej z literatury \cite{costaBook}, \cite{gali_gov_spending} oraz \cite{gali}:
        \begin{center}
            \begin{tabular}{|p{0.2\textwidth}|p{0.2\textwidth}|}
                \hline
                Parametr & Wartość \\
                \hline 
                $\sigma$ & $2$\\
                $\phi$ & $1.5$ \\
                $\alpha$ & $0.35$ \\
                $\beta$ & $0.985$ \\
                $\delta$ & $0.035$ \\
                $\rho_A$ & $0.95$ \\
                $\psi$ & $3$ \\
                $\theta$ & $0.75$ \\
                $\psi_1$ & $1.5$ \\
                $\psi_2$ & $0$ \\
                $\rho_Q$ & $0.015$ \\
                \hline 
            \end{tabular} 
         \end{center}
    \item Kalibracji szoków (szok impulsowy):
    \begin{gather}
        \epsilon_{A,t} =
        \begin{cases}
            0.1 & \text{dla $t = 1$,} \\
            0 & \text{dla $t > 1$.}
        \end{cases}
    \end{gather}
\end{enumerate}

\section{Opis implementacji}
\label{sec:implementation_result}

Implementacja została wykonana w języku Python w wersji 3.9. Decyzja na wykorzystanie tej technologii motywowana była dostępnością wielu bibliotek numerycznych oraz parserów równań, które ułatwią szeroko stosowane obliczenia macierzowe wykorzystywane w rozwiązaniach. Dodatkowo język pozwala na zwięzłe przedstawienie idei wykorzystanych metod. Zastosowane biblioteki zostały opisane w sekcji \ref{sec:used_libraries}. Dodatkowo w celu porównania rozwiązań została użyta platforma Dynare, udostępniająca implementacje narzędzi ekonomicznych. Do przygotowanej solucji zostały dołączone modele z rozszerzeniem \emph{.mod}, które mogą zostać uruchomione na wspomnianej platformie. 

Repozytorium z kodem dostępne jest pod adresem \url{https://github.com/wgrabis/DSGE}. Platforma Dynare dostępna jest pod adresem \url{https://www.dynare.org/}. Uruchomienie programu odbywa się poprzez komendę:
\begin{equation}
    \texttt{program <model-file> [optional args]} \nonumber
\end{equation}
z następującymi argumentami:
\begin{center}
    \begin{tabular}{p{0.32\textwidth}p{0.58\textwidth}}
          \texttt{-m}/\texttt{--mode} & Tryb uruchomienia programu dopuszczalne wartości to \emph{fblanchard}, \emph{estimate} lub \emph{festimate}. \\
         \texttt{-d}/\texttt{--debug} & Ustawia flagę dla logowania dodatkowych informacji.\\
         \texttt{-t}/\texttt{--time} [liczba] & Czas prognozowania modelu w postaci liczby naturalnej, domyślnie $40$. \\
         \texttt{-r}/\texttt{--rounds} [liczba] & Ilość wykonanych rund w algorytmie estymacji, domyślnie $100$. \\
         \texttt{-sp}/\texttt{--singlePlot} & Opcja włączająca wyświetlanie wykresów dla wszystkich zmiennych w jednym oknie. \\
         \texttt{-pdir}/\texttt{--plotDir} [ścieżka] & Ścieżka do zapisu dla wygenerowanych wykresów w formacie \emph{.png}. \\
         \texttt{-ds}/\texttt{--disableShow}  & Wyłączenie opcji renderowania okien z wykresami. \\
         \texttt{-ra}/\texttt{--runAgainst} [ścieżka]  & Uruchamia drugi plik modelu z tymi samymi argumentami i nakłada wyniki na wykresy obecnego modelu, dostępne jedynie dla opcji \emph{fblanchard}.  \\
    \end{tabular}
\end{center}
W zależności od wykorzystanej opcji \texttt{--mode} program zostanie uruchomiony jako:
\begin{itemize}
    \item \emph{fblanchard} - zostanie wykonana prognoza szoku impulsowego dla dostępnej kalibracji lub wartości oczekiwanych parametrów dla czasu \texttt{--time},
    \item \emph{estimate} - zostanie przeprowadzona estymacja dla \texttt{--rounds} rund i zostaną wygenerowane wykresy dla przebiegu zaakceptowanej próbki w algorytmie Metropolisa-Hastingsa,
    \item \emph{festimate} - zostaną wyświetlone wykresy dla estymacji jak w przypadku opcji \emph{estimate} oraz dodatkowo wykresy dla prognozy metodą ścieżek losowych dla czasu \texttt{--time}.
\end{itemize}
W przypadku uruchomienia bez opcji \texttt{--disableShow} program powinien zakończyć się pokazaniem wykresów dla zmiennych w kontekście odpalonego trybu (prognozy zmiennych obserwowanych lub prognozy parametrów).

\subsection{Struktura repozytorium}

Repozytorium podzielone jest na następujące katalogi:
\begin{itemize}
    \item \fpath{/src} -- katalog zawierający kod wykorzystywany w implementacji,
    \item \fpath{/samples} -- folder z przykładowymi modelami, wśród modeli najbardziej istotnymi są:
        \begin{itemize}
            \item \fpath{rbcModel.json} -- plik z przykładowym modelem RBC,
            \item \fpath{nekModel.json} -- plik z modelem ekonomii nowokeynesowskiej dla szoku produktywności,
            \item \fpath{nekModelMonetary.json} -- plik z modelem ekonomii nowokeynesowskiej dla szoku polityki monetarnej,
            \item \fpath{philipCurve.json} -- plik z modelem przedstawiającym krzywą Philipsa,
            \item \fpath{ireland.json} -- plik z modelem przedstawiającym gospodarkę Irlandii,
        \end{itemize}
    \item \fpath{/dynare} -- folder zawierający przykładowe modele do wykorzystania w Dynare w celu porównania wyników implementacji z repozytorium z implementacją Dynare,
    \item \fpath{/paper} -- folder zawierający tekst pracy magisterskiej.
\end{itemize}

Katalog zawierający kod podzielony jest na następujące części:
\begin{itemize}
    \item \fpath{/filter} -- pakiet kodu z implementacją dla algorytmu filtru wykorzystywanego w funkcji wiarygodności,
    \item \fpath{/forecast} -- zawiera implementację prognozowania szoku impulsowego dla kalibracji oraz prognozowania losowymi ścieżkami,
    \item \fpath{/format} -- implementacja parsowania plików modelu oraz danych estymowania,
    \item \fpath{/helper} -- dodatkowe pomocnicze funkcjonalności m.in. wyświetlania wykresów przy pomocy biblioteki \emph{matplotlib},
    \item \fpath{/likelihood} -- pakiet zawierający implementację funkcji wiarygodności w oparciu o dostarczony algorytm filtrowania,
    \item \fpath{/metropolis\_hastings} -- katalog zawierający implementacje algorytmu Metropolisa-Hastingsa,
    \item \fpath{/model} -- zbiór implementacji pozwalający na budowanie modelu w oparciu o sparsowane dane oraz pomocnicze klasy służące do reprezentacji funkcji pomiaru oraz funkcji macierzy,
    \item \fpath{/solver} -- zawiera implementację metody Blancharda Kahna oraz uogólnionej metody rozwiązywania modeli,
    \item \fpath{/util} -- pomocniczy zestaw klas dla parametrów uruchomienia programu.
\end{itemize}

% + opisać pakiety kodu

\subsection{Wykorzystane biblioteki}
\label{sec:used_libraries}

\begin{itemize}
    \item \libname{Numpy} -- szeroko wykorzystywana biblioteka udostępniająca reprezentację wektorów i macierzy oraz szereg metod numerycznych.
    \item \libname{Sympy} -- biblioteka wykorzystywana do:
        \begin{itemize}
            \item parsowania równań -- równania w plikach modeli są parsowane przy pomocy \emph{Sympy} do postaci macierzy zmiennych, które następnie mogą być uzupełnione wartościami w celu wygenerowania macierzy liczb rzeczywistych, przykładowo udostępnia metody tworzenia macierzy równań liniowych postaci:
            \begin{equation}
                A \begin{bmatrix}
                    x_1 \\
                    x_2 \\
                    \dots \\
                    x_n
                \end{bmatrix} = 0,
            \end{equation}
            \item dodatkowe operacje na macierzach -- implementacja macierzy w \emph{Sympy} pozwala na zmianę na odpowiednią reprezentację w \emph{Numpy}, dzięki czemu można wykorzystywać przemiennie metody z obu bibliotek w celach obliczeń macierzowych,
        \end{itemize}
    \item \libname{Scipy} -- biblioteka udostępniająca implementację pewnych operacji macierzowych wykorzystywanych przy modelach, brakujących w pozostałych bibliotekach np. dekompozycję QZ,
    \item \libname{Matplotlib} -- biblioteka udostępniająca metody generowania wykresów w Pythonie, wykorzystywana do wyświetlania wyników prognozowania dla modeli RBC i DSGE.
\end{itemize}

\subsection{Plik modelu}

W tej części zostanie omówiony plik modelu wykorzystywany przez solucję. Poniższa struktura przestawia model zaprezentowany w sekcji \ref{sec:full_linear_model_sample}. Plik składa się z następujących części:
\begin{itemize}
    \item Nazwa modelu (\emph{name}):
    \begin{lstlisting}[language=json,firstnumber=1]
 "name": "nek model simple monetary"
    \end{lstlisting}
    
    \item Zmienne stanu (\emph{variables}) -- zmienne makroekonomiczne występujące w opisywanej gospodarce. Na tym etapie nie ma potrzeby rozdzielenia na zmienne stanu i kontrolne, ten podział odbywa się na bazie równań modelu:
    \begin{lstlisting}[language=json,firstnumber=2]
 "variables": ["YV", "IV", "CV", "RK", "KV", "WV", "LV", "MC", "PI", "AV", "RQ"]
    \end{lstlisting}
    \item Zmienne strukturalne (\emph{structural}) -- parametry występujące w równaniach modelu, które są analogicznie przedstawione w postaci listy zmiennych:
    \begin{lstlisting}[language=json,firstnumber=3]
 "structural": ["sigmaSt", "phiSt", "alphaSt", "betaSt", "deltaSt", "rhoASt", "psiSt", "thetaSt", "phiPSt", "rhoQSt"]
    \end{lstlisting}
    W przypadku wykorzystania estymacji, reprezentacja dla pojedynczej zmiennej może zostać rozszerzona do postaci:
    \begin{lstlisting}[language=json,firstnumber=2]
{
  "name": "rho_g",
  "display": "\\rho_g"
},
    \end{lstlisting}
    Nazwa zmiennej \emph{name} zostanie wykorzystana w równaniach oraz do nazwania pliku \emph{.png} z wykresem, natomiast \emph{display} zostanie wygenerowana na wykresie wykonanym przy wykorzystaniu rozszerzenia \libname{Matplotlib} przez \LaTeX.
    \item Zmienne egzogeniczne (\emph{shocks}) -- zmienne odpowiadające szokom przekazane w postaci listy nazw zmiennych:
    \begin{lstlisting}[language=json,firstnumber=4]
 "shocks": ["eA"]
    \end{lstlisting}
    
    \item Równania modelu (\emph{model}) -- podzielone na poszczególne kategorie równań:
    \begin{lstlisting}[language=json,firstnumber=5]
 "model": {
    "definitions": [...],
    "equations": [...],
    "observables": [...]
 },
    \end{lstlisting}
        \begin{itemize}
            \item Równania opisujące stan ustalony (\emph{definitions}) -- równania opisujące dodatkowe zmienne dla stanu ustalonego, w celu uproszczenia definicji dla modelu:
            \begin{lstlisting}[language=json,firstnumber=6]
 "definitions": [
    "Rss = (1/betaSt)-(1-deltaSt)",
    "MCss = (psiSt-1)/psiSt",
    "Wss = (1-alphaSt)*(MCss*((alphaSt/Rss)^alphaSt))^(1/(1-alphaSt))",
    "Yss = ((Rss/(Rss-deltaSt*alphaSt*MCss))^(sigmaSt/(sigmaSt+phiSt)))*((Wss)*(Wss/((1-alphaSt)*MCss))^phiSt)^(1/(sigmaSt+phiSt))",
    "Kss = alphaSt*MCss*(Yss/Rss)",
    "Iss = deltaSt*Kss",
    "Css = (Wss*(Wss/(Yss*(1-alphaSt)*MCss))^phiSt)^(1/sigmaSt)",
    "Lss = (1-alphaSt)*MCss*(Yss/Wss)"
 ]
            \end{lstlisting}
            \item Równania modelu (\emph{equations}) -- równania w postaci liniowej opisujące stan równowagi opisywanej gospodarki, zmienne występujące w modelu mogą mieć postać:
            \begin{itemize}
                \item zmienne kontrolne z indeksem $(+1)$ np. $C(+1)$,
                \item zmienne obecnego okresu bez indeksu np. $C$,
                \item zmienne poprzedniego okresu z indeksem $(-k)$ np. $A(-1), A(-3)$.
            \end{itemize}
            \begin{lstlisting}[language=json,firstnumber=16]
 "equations": [
    "sigmaSt*CV + phiSt*LV = WV",
    "(sigmaSt/betaSt)*(CV(+1)-CV)=Rss*RK(+1)",
    "KV = (1-deltaSt)*KV(-1) + deltaSt * IV",
    "sigmaSt*(CV(+1) - CV) = RQ - PI(+1)",
    "KV(-1) = MC + YV - RK",
    "LV = MC + YV - WV",
    "YV = AV + alphaSt*KV(-1) + (1-alphaSt)*LV",
    "PI = betaSt*PI(+1)+((1-thetaSt)*(1-betaSt*thetaSt)/thetaSt)*MC",
    "AV = rhoASt*AV(-1) + eA",
    "Yss*YV = Css*CV + Iss*IV",
    "RQ=(1-rhoQSt)*(1 + phiPSt * PI)  + rhoQSt * RQ(-1)"
 ],
            \end{lstlisting}
            
            
            \item Równania definiujące zmienne obserwowalne (\emph{observables}) -- równania w postaci:
            \begin{equation}
                Y_{a} = f(t) + h(s_t) + g,
            \end{equation}
            gdzie funkcja $f$ reprezentuje zależność zmiennej obserwowalnej od obecnego czasu $t$, a $h$ służy mapowaniu wartości wektora $s_t$ na wartość zmiennej obserwowalnej i jako ostatnia g jest stałą. W przykładowym modelu zostaną wykorzystane jedynie proste definicje w celu przygotowania wykresów dla wszystkich zmiennych:
            \begin{lstlisting}[language=json,firstnumber=29]
"observables": [
    "OL = LV",
    "OC = CV",
    "OPI = PI",
    "OK = KV",
    "OY = YV",
    "OMC = MC",
    "OIV = IV",
    "OW = WV",
    "ORQ = RQ",
    "ORK = RK",
    "OA = AV"
]
            \end{lstlisting}
        \end{itemize}
    \item Rozkład a priori (\emph{priors}) -- rozkład prawdopodobieńśtwa dla parametrów modelu (w przypadku szoków rozkład opisuje postać szoków dla obecnego momentu):
    \begin{lstlisting}[language=json,firstnumber=43]
 "priors": {
    "eA": {
        "distribution": "normal",
        "mean": 1,
        "variance": 0.1
    },
    "sigmaSt": {
        "distribution": "normal",
        "mean": 2,
        "variance": 0.01
    },
        ...
 },
    \end{lstlisting}
    Poszczególne elementy są opisane w jeden z następujących sposobów:
    \begin{itemize}
        \item jako parametry kalibrowane:
            \begin{lstlisting}[language=json,firstnumber=43]
"betaSt": {
  "distribution": "calibration",
  "value": 0.99
},
            \end{lstlisting}
        \item jako parametry określone przez odpowiedni rozkład prawdopodobieństwa z dodatkową specyfikacją ograniczeń górnych i dolnych dla zmiennej:
            \begin{lstlisting}[language=json,firstnumber=43]
"alpha_x": {
  "distribution": "normal",
  "mean": 0.2028,
  "variance": 0.1,
  "lowerBound": 0,
  "upperBound": 1
},
            \end{lstlisting}        
    \end{itemize}
    \item Dane do estymacji modelu (\emph{estimations}) -- opisane w osobnym pliku \emph{.csv} lub w postaci tablicy danych. Każda pozycja dla estymowanych danych musi definiować tyle wartości ile zmiennych obserwowalnych zdefiniowanych w pliku w sekcji \emph{observables}. Model z sekcji \ref{sec:full_linear_model_sample} został jedynie przygotowany pod prognozowanie, stąd poniższa reprezentacja pochodzi z innego modelu i nie odpowiada przykładowym wartością obserwowalnym:
    \begin{lstlisting}[language=json,firstnumber=1]
"estimations": {
    "data": [
        ["1.0", "1.0"],
        ["2.0", "1.5"],
        ["1.0", "1.0"],
        ["1.5", "1.25"]
    ]
}
    \end{lstlisting}
    W celu specyfikacji pliku z danymi estymacji stosujemy poniższą reprezentację:
    \begin{lstlisting}[language=json,firstnumber=1]
"estimations": {
    "name": "samples/ireland_data.csv"
}
    \end{lstlisting}
\end{itemize}

\section{Wyniki prognozowania i estymacji}

W tej części pracy zostaną przedstawione wyniki dla zaprezentowanego modelu oraz pozostałych wykorzystanych w trakcie implementacji. Rozdział \ref{sec:results_nek_prod} opisuje wyniki modelu nowokeynesowskiego dla impulsowego szoku produktywności. Kolejna sekcja \ref{sec:results_nek_mon} poświęcona jest wynikom tego samego modelu w reakcji na impulsowy szok monetarny. W rozdziale \ref{sec:results_rbc_prod} przedstawione jest porównanie wyników szoku impulsowego z prostym modelem RBC w celu opisania różnic obu teorii. 

\subsection{Wyniki prognozowania dla szoku produktywności w modelu NEK}
\label{sec:results_nek_prod}

Wykonano symulację prognoz modelu dla szoku impulsowego dla wartości wskazanej w kalibracji sekcji \ref{sec:full_linear_model_sample}. Poniżej zaprezentowane zostały wykresy dla prognozowanych zmiennych:
\begin{center}
    \begin{minipage}{.3\textwidth}
      \centering
      \captionsetup{type=figure}
      \includegraphics[width=.8\linewidth]{images/nek/L.png}
      \captionof{figure}{Czas pracy}
      \label{fig:nek:L}
    \end{minipage}%
    \begin{minipage}{.3\textwidth}
      \centering
      \captionsetup{type=figure}
      \includegraphics[width=.8\linewidth]{images/nek/C.png}
      \captionof{figure}{Konsumpcja}
      \label{fig:nek:C}
    \end{minipage}
    \begin{minipage}{.3\textwidth}
      \centering
      \captionsetup{type=figure}
      \includegraphics[width=.8\linewidth]{images/nek/PI.png}
      \captionof{figure}{Inflacja}
      \label{fig:nek:PI}
    \end{minipage}
\end{center}

\begin{center}
    \begin{minipage}{.3\textwidth}
      \centering
      \captionsetup{type=figure}
      \includegraphics[width=.8\linewidth]{images/nek/K.png}
      \captionof{figure}{Kapitał}
      \label{fig:nek:K}
    \end{minipage}%
    \begin{minipage}{.3\textwidth}
      \centering
      \captionsetup{type=figure}
      \includegraphics[width=.8\linewidth]{images/nek/Y.png}
      \captionof{figure}{Produkcja}
      \label{fig:nek:Y}
    \end{minipage}
    \begin{minipage}{.3\textwidth}
      \centering
      \captionsetup{type=figure}
      \includegraphics[width=.8\linewidth]{images/nek/MC.png}
      \captionof{figure}{Koszt krańcowy}
      \label{fig:nek:MC}
    \end{minipage}
\end{center}

\begin{center}
    \begin{minipage}{.3\textwidth}
      \centering
      \captionsetup{type=figure}
      \includegraphics[width=.8\linewidth]{images/nek/I.png}
      \captionof{figure}{Inwestycje}
      \label{fig:nek:I}
    \end{minipage}%
    \begin{minipage}{.3\textwidth}
      \centering
      \captionsetup{type=figure}
      \includegraphics[width=.8\linewidth]{images/nek/W.png}
      \captionof{figure}{Realna płaca}
      \label{fig:nek:W}
    \end{minipage}
    \begin{minipage}{.3\textwidth}
      \centering
      \captionsetup{type=figure}
      \includegraphics[width=.8\linewidth]{images/nek/RQ.png}
      \captionof{figure}{Stopa procentowa}
      \label{fig:nek:RQ}
    \end{minipage}
\end{center}

\begin{center}
    \begin{minipage}{.4\textwidth}
      \centering
      \captionsetup{type=figure}
      \includegraphics[width=.8\linewidth]{images/nek/RK.png}
      \captionof{figure}{Zwrot z kapitału}
      \label{fig:nek:RK}
    \end{minipage}%
    \begin{minipage}{.4\textwidth}
      \centering
      \captionsetup{type=figure}
      \includegraphics[width=.8\linewidth]{images/nek/A.png}
      \captionof{figure}{Produktywność}
      \label{fig:nek:A}
    \end{minipage}
\end{center}

Skutkiem szoku produktywności jest wzrost konsumpcji, produkcji, inwestycji oraz płac realnych w czasie zwiększonej produktywności. Gospodarstwa domowe reagują zwiększając konsumpcję oraz inwestycję w celu zbudowania kapitału. Chwilowo czas pracy jest zwiększony przez nagły wzrost zmiennych, jednak z czasem pasywny wpływ z większego kapitał prowadzi do większej użyteczności, stąd zmniejszenie czasu pracy poniżej oryginalnego poziomu. Krańcowa użyteczność konsumpcji względem czasu wolnego spada, ze względu na zwiększone zarobki, a gospodarstwa zarabiając tyle samu w mniejszym czasie pracy. Inflacja gospodarki początkowo wzrasta zgodnie ze wzrostem ilości dóbr na rynku, jednak z czasem spadku pracy oraz dużej ilości towarów na rynku odnotowujemy spadek cen w kolejnych okresach. Zmiana inflacji odpowiada korekcji na stopę procentową zgodnie z przyjętą regułą polityki monetarnej. Wraz ze wzrostem kapitału początkowy zwrot z kapitału spada do wartości poniżej stanu ustalonego, a co za tym idzie spadek zysku z inwestycji gospodarstw domowych. Ostatecznie szok produktywności prowadzi do wzrostu dla produkcji i użyteczności.

\subsection{Wyniki prognozowania dla szoku monetarnego w modelu NEK}
\label{sec:results_nek_mon}

Dla wybranego modelu przeanalizowano również szok monetarny. Wprowadzenie go do modelu odbyło się poprzez zmianę równania polityki monetarnej \eqref{eq:monetary_nek_full_final}:
\begin{equation}
    r^q_t = (1-\rho_Q)r^{q^*}_t + \rho_Q r^q_{t-1} + \epsilon_{M,t},
\end{equation}
gdzie zmienna $\epsilon_{M,t}$ opisuje szok polityki monetarnej z kalibracją:
\begin{gather}
        \epsilon_{M,t} =
        \begin{cases}
            0.01 & \text{dla $t = 1$,} \\
            0 & \text{dla $t > 1$.}
        \end{cases}
\end{gather}
Poniższe wykresy prezentują rezultat prognozowania:
\begin{center}
    \begin{minipage}{.3\textwidth}
      \centering
      \captionsetup{type=figure}
      \includegraphics[width=.8\linewidth]{images/nek-monetary/L.png}
      \captionof{figure}{Czas pracy}
      \label{fig:nek-monetary:L}
    \end{minipage}%
    \begin{minipage}{.3\textwidth}
      \centering
      \captionsetup{type=figure}
      \includegraphics[width=.8\linewidth]{images/nek-monetary/C.png}
      \captionof{figure}{Konsumpcja}
      \label{fig:nek-monetary:C}
    \end{minipage}
    \begin{minipage}{.3\textwidth}
      \centering
      \captionsetup{type=figure}
      \includegraphics[width=.8\linewidth]{images/nek-monetary/PI.png}
      \captionof{figure}{Inflacja}
      \label{fig:nek-monetary:PI}
    \end{minipage}
\end{center}

\begin{center}
    \begin{minipage}{.3\textwidth}
      \centering
      \captionsetup{type=figure}
      \includegraphics[width=.8\linewidth]{images/nek-monetary/K.png}
      \captionof{figure}{Kapitał}
      \label{fig:nek-monetary:K}
    \end{minipage}%
    \begin{minipage}{.3\textwidth}
      \centering
      \captionsetup{type=figure}
      \includegraphics[width=.8\linewidth]{images/nek-monetary/Y.png}
      \captionof{figure}{Produkcja}
      \label{fig:nek-monetary:Y}
    \end{minipage}
    \begin{minipage}{.3\textwidth}
      \centering
      \captionsetup{type=figure}
      \includegraphics[width=.8\linewidth]{images/nek-monetary/MC.png}
      \captionof{figure}{Koszt krańcowy}
      \label{fig:nek-monetary:MC}
    \end{minipage}
\end{center}

\begin{center}
    \begin{minipage}{.3\textwidth}
      \centering
      \captionsetup{type=figure}
      \includegraphics[width=.8\linewidth]{images/nek-monetary/I.png}
      \captionof{figure}{Inwestycje}
      \label{fig:nek-monetary:I}
    \end{minipage}%
    \begin{minipage}{.3\textwidth}
      \centering
      \captionsetup{type=figure}
      \includegraphics[width=.8\linewidth]{images/nek-monetary/W.png}
      \captionof{figure}{Realna płaca}
      \label{fig:nek-monetary:W}
    \end{minipage}
    \begin{minipage}{.3\textwidth}
      \centering
      \captionsetup{type=figure}
      \includegraphics[width=.8\linewidth]{images/nek-monetary/RQ.png}
      \captionof{figure}{Stopa procentowa}
      \label{fig:nek-monetary:RQ}
    \end{minipage}
\end{center}

\begin{center}
    \begin{minipage}{.4\textwidth}
      \centering
      \captionsetup{type=figure}
      \includegraphics[width=.8\linewidth]{images/nek-monetary/RK.png}
      \captionof{figure}{Zwrot z kapitału}
      \label{fig:nek-monetary:RK}
    \end{minipage}%
    \begin{minipage}{.4\textwidth}
      \centering
      \captionsetup{type=figure}
      \includegraphics[width=.8\linewidth]{images/nek-monetary/A.png}
      \captionof{figure}{Produktywność}
      \label{fig:nek-monetary:A}
    \end{minipage}
\end{center}

Pierwszym z wniosków jaki nasuwają wyniki wpływu szoku monetarnego jest spadek wielu zmiennych w pierwszym okresie ze względu na nagłą zmianę stopy procentowej. Możemy zauważyć, że w reakcji na zdarzenie gospodarstwa domowe obniżają czas pracy, konsumpcję oraz inwestycje. W przypadku konsumpcji negatywna zmiana utrzymuje się długofalowo w odpowiedzi do podwyższonej stopy procentowej. Jedną z interpretacji tego faktu może być nagły spadek cen obligacji skarbowych, które okazują się dobrym źródłem lokowania zarobków i pozytywnego zysku, jako że model przedstawia wartość kosztu obligacji jako $\frac{1}{R_t^Q}$. Ta zmiana nie utrzymuje się w przypadku inwestycji, gdzie zwrot z kapitału powraca do normy po początkowym impulsowym spadku. Niestety deprecjacja kapitału sprawia, że powraca on do stanu równowagi w podobnym tempie jak konsumpcja oraz stopa procentowa. W przypadku firm szok monetarny nie wpływa na produktywność, stąd po impulsowym spadku produkcji sytuacja powraca do równowagi w kolejnych okresach, wszelkie długofalowe zmiany są ograniczone w problemie użyteczności gospodarstw domowych.

\subsection{Wyniki prognozowania dla szoku produktywności w modelu RBC}
\label{sec:results_rbc_prod}

W rozdziale \ref{sec:rbc_model_sample} przedstawiono model RBC. W tej części zostanie przytoczona pełna postać tego modelu po przeprowadzeniu linearyzacji. Skalibrowane wartości parametrów odpowiadają modelowi nowokeynesowskiemu \ref{sec:full_linear_model_sample}.

Model prezentuje się następująco:
\begin{enumerate}
    \item Stan ustalony modelu:
        \begin{align}
                \steady{A} &= 1, \\
                \steady{R}^K &= \frac{1}{\beta} - (1 - \delta), \\
                \steady{W} &= (1-\alpha)\left(\steady{A} \left(\frac{\alpha}{\steady{R}^{K}}\right)^\alpha\right)^{\frac{1}{1-\alpha}}, \\
                \steady{Y} &= \left( \steady{W} \left( \frac{\steady{W}}{1-\alpha}\right)^\phi\right)^{\frac{1}{\sigma + \phi}} \frac{\steady{R}}{\steady{R} - \delta \alpha}^{\frac{\sigma}{\sigma + \phi}}, \\
                \steady{L} &= (1-\alpha)\frac{\steady{Y}}{\steady{W}}, \\
                \steady{K} &= \alpha\frac{\steady{Y}}{\steady{R}^K},\\
                \steady{I} &= \delta \steady{K}, \\
                \steady{C} &= \steady{Y} - \steady{I}.
        \end{align}
    \item Równania modelu:
    \begin{gather}
        \sigma c_{t} + \phi l_t = w \\
        \frac{\sigma}{\beta} (\E_t c_{t+1} - c_t) = \steady{R}^K \E_t r^K_{t+1} \\
        k_t = (1 - \delta) k_{t-1} + \delta i_t \\ 
        y_t = a_t + (1-\alpha) l_t + \alpha k_{t-1} \\
        r^K_t = y_t - k_{t-1}   \\
        w_t = y_t - l_t \\
        \steady{Y} y_t = \steady{C} c_t + \steady{I} i_t \\
        a_t = \rho a_{t-1} + \epsilon_{A,t}
    \end{gather}
    \item Kalibracja parametrów:
        \begin{center}
            \begin{tabular}{|p{0.2\textwidth}|p{0.2\textwidth}|}
                \hline
                Parametr & Wartość \\
                \hline 
                $\sigma$ & $2$\\
                $\phi$ & $1.5$ \\
                $\alpha$ & $0.35$ \\
                $\beta$ & $0.985$ \\
                $\delta$ & $0.035$ \\
                $\rho_A$ & $0.95$ \\
                \hline 
            \end{tabular} 
         \end{center}
    \item Kalibracja szoków (szok impulsowy):
    \begin{gather}
        \epsilon_{A,t} =
        \begin{cases}
            0.1 & \text{dla $t = 1$,} \\
            0 & \text{dla $t > 1$.}
        \end{cases}
    \end{gather}
\end{enumerate}

Rezultaty prognozowania zostały wykonane ponownie dla 40 okresów. Na wykresach czerwoną linią zostały przedstawione wyniki modelu RBC, natomiast przerywaną linią zostały naniesione dla porównania wyniki modelu nowokeynesowskiego \ref{sec:full_linear_model_sample}. 

\begin{center}
    \begin{minipage}{.3\textwidth}
      \centering
      \captionsetup{type=figure}
      \includegraphics[width=.8\linewidth]{images/rbc/L.png}
      \captionof{figure}{Czas pracy}
      \label{fig:rbc:L}
    \end{minipage}%
    \begin{minipage}{.3\textwidth}
      \centering
      \captionsetup{type=figure}
      \includegraphics[width=.8\linewidth]{images/rbc/C.png}
      \captionof{figure}{Konsumpcja}
      \label{fig:rbc:C}
    \end{minipage}
    \begin{minipage}{.3\textwidth}
      \centering
      \captionsetup{type=figure}
      \includegraphics[width=.8\linewidth]{images/rbc/K.png}
      \captionof{figure}{Kapitał}
      \label{fig:rbc:K}
    \end{minipage}
\end{center}

\begin{center}
    \begin{minipage}{.3\textwidth}
      \centering
      \captionsetup{type=figure}
      \includegraphics[width=.8\linewidth]{images/rbc/Y.png}
      \captionof{figure}{Produkcja}
      \label{fig:rbc:Y}
    \end{minipage}%
    \begin{minipage}{.3\textwidth}
      \centering
      \captionsetup{type=figure}
      \includegraphics[width=.8\linewidth]{images/rbc/I.png}
      \captionof{figure}{Inwestycje}
      \label{fig:rbc:I}
    \end{minipage}
    \begin{minipage}{.3\textwidth}
      \centering
      \captionsetup{type=figure}
      \includegraphics[width=.8\linewidth]{images/rbc/W.png}
      \captionof{figure}{Realna płaca}
      \label{fig:rbc:W}
    \end{minipage}
\end{center}

\begin{center}
    \begin{minipage}{.4\textwidth}
      \centering
      \captionsetup{type=figure}
      \includegraphics[width=.8\linewidth]{images/rbc/RK.png}
      \captionof{figure}{Zwrot z kapitału}
      \label{fig:rbc:RK}
    \end{minipage}%
    \begin{minipage}{.4\textwidth}
      \centering
      \captionsetup{type=figure}
      \includegraphics[width=.8\linewidth]{images/rbc/A.png}
      \captionof{figure}{Produktywność}
      \label{fig:rbc:A}
    \end{minipage}
\end{center}
Możemy zauważyć podobną reakcję zmiennych modelu na wystąpienie szoku produktywności co w modelu NEK, jednak w przypadku modelu z konkurencją doskonałą odchylenia wartości zmiennych od stanu ustalonego są mniejsze. Z czasem zjawiska sztywności modelu nowokeynesowskiego wygasają i wartości obu modeli zbiegają do podobnego poziomu przed powrotem do stanu ustalonego. Produkcja w obu gospodarkach jest niemalże identyczna z ograniczeniem do potencjalnie innej wartości stanu ustalonego w zależności od wartości parametrów modelu.

\subsection{Wyniki estymacji bayesowskiej dla modelu gospodarki Irlandii}

W celu zaprezentowania wyników estymacji zostanie wykorzystany model Irlandii z pracy \cite{NBERw10309} opracowany w repozytorium \cite{Pfeifer_DSGE_mod_A_collection} razem z danymi estymacji oraz rozkładem a priori dla przedziału od 1940 do 2004 roku.

Model prezentuje się następująco:
\begin{enumerate}
    \item Równania modelu:
    \begin{gather}
        a_{t} = \rho_A a_{t-1} + \epsilon_{A,t},\\
        e_{t} = \rho_E e_{t-1} + \epsilon_{E,t},\\
        z_{t} = \epsilon_{Z,t},\\
        x_{t} = \alpha_X x_{t-1} + (1 - \alpha_X)\E_t x_{t+1} - (\hat{r} - \E_t \hat{\pi}_{t+1}) + (1-\omega) (1-\rho_A)a_{t},\\
        \hat{\pi}_{t} = \beta(\alpha_\pi \hat{\pi}_{t-1} + (1-\alpha_\pi)\E_t \hat{\pi}_{t+1} + \psi x_{t} - e_{t},\\
        x_{t} = \hat{y}_t - \omega a_{t},\\
        \hat{g}_{t} = \hat{y}_{t} - \hat{y}_{t-1} + z_t,\\
        \hat{r}_t - \hat{r}_{t-1} = \rho_\pi \hat{\pi}_t + \rho_g \hat{g}_t + \rho_X x_t + \epsilon_{R,t}.
    \end{gather}
    \item Zmienne obserwowalne:
    \begin{gather}
        g_{obs} = \hat{g}_t,\\
        r_{obs} = \hat{r}_t,\\
        \pi_{obs} = \hat{\pi}_t.
    \end{gather}
    \item Kalibracja parametrów:
        \begin{center}
            \begin{tabular}{|p{0.2\textwidth}|p{0.2\textwidth}|}
                \hline
                Parametr & Wartość \\
                \hline 
                $\beta$ & $0.99$\\
                $\psi$ & $0.99$ \\
                \hline 
            \end{tabular} 
         \end{center}
    \item Rozkład parametrów:
        \begin{center}
            \begin{tabular}{|p{0.2\textwidth}|p{0.2\textwidth}|p{0.2\textwidth}|}
                \hline
                Parametr & A priori & Ograniczenia\\
                \hline 
                $\omega$ & $0.00001$ & brak\\
                $\alpha_X$ & $0.2028$ & $[0,1]$\\
                $\alpha_\pi$ & $0.00001$ & $[0,1]$ \\
                $\rho_g$ & $0.2365$ & $[0,1]$ \\
                $\rho_\pi$ & $0.3053$ & $[0,1]$ \\
                $\rho_X$ & $0.00001$ & $[0,1]$ \\
                $\rho_A$ & $0.9910$ & $[0,1]$ \\
                $\rho_E$ & $0.5439$ & $[0,1]$ \\
                \hline 
            \end{tabular} 
         \end{center}
    \item Rozkład szoków:
    \begin{gather}
        \epsilon_{A,t} \sim  \mathcal{N}(0, 0.000912),\\
        \epsilon_{E,t} \sim  \mathcal{N}(0, 0.000009),\\
        \epsilon_{Z,t} \sim  \mathcal{N}(0, 0.000079),\\
        \epsilon_{R,t} \sim  \mathcal{N}(0, 0.000008)
    \end{gather}
\end{enumerate}

Przebieg zaakceptowanej przez algorytm próbki prezentuje się następująco:

\begin{center}
    \begin{minipage}{.3\textwidth}
      \centering
      \captionsetup{type=figure}
      \includegraphics[width=.8\linewidth]{images/estimations/omegaSt.png}
      \captionof{figure}{$\omega$}
      \label{fig:estimation:omega}
    \end{minipage}%
    \begin{minipage}{.3\textwidth}
      \centering
      \captionsetup{type=figure}
      \includegraphics[width=.8\linewidth]{images/estimations/alpha_x.png}
      \captionof{figure}{$\alpha_X$}
      \label{fig:estimation:alpha_x}
    \end{minipage}
    \begin{minipage}{.3\textwidth}
      \centering
      \captionsetup{type=figure}
      \includegraphics[width=.8\linewidth]{images/estimations/alpha_pi.png}
      \captionof{figure}{$\alpha_{\pi}$}
      \label{fig:estimation:alpha_pi}
    \end{minipage}
\end{center}

\begin{center}
    \begin{minipage}{.3\textwidth}
      \centering
      \captionsetup{type=figure}
      \includegraphics[width=.8\linewidth]{images/estimations/rho_g.png}
      \captionof{figure}{$\rho_g$}
      \label{fig:estimation:rho_g}
    \end{minipage}%
    \begin{minipage}{.3\textwidth}
      \centering
      \captionsetup{type=figure}
      \includegraphics[width=.8\linewidth]{images/estimations/rho_pi.png}
      \captionof{figure}{$\rho_{\pi}$}
      \label{fig:estimation:rho_pi}
    \end{minipage}
    \begin{minipage}{.3\textwidth}
      \centering
      \captionsetup{type=figure}
      \includegraphics[width=.8\linewidth]{images/estimations/rho_x.png}
      \captionof{figure}{$\rho_X$}
      \label{fig:estimation:rho_x}
    \end{minipage}
\end{center}

\begin{center}
    \begin{minipage}{.4\textwidth}
      \centering
      \captionsetup{type=figure}
      \includegraphics[width=.8\linewidth]{images/estimations/rho_a.png}
      \captionof{figure}{$\rho_A$}
      \label{fig:estimation:rho_A}
    \end{minipage}%
    \begin{minipage}{.4\textwidth}
      \centering
      \captionsetup{type=figure}
      \includegraphics[width=.8\linewidth]{images/estimations/rho_e.png}
      \captionof{figure}{$\rho_E$}
      \label{fig:estimation:rho_E}
    \end{minipage}
\end{center}

Estymacja została wykonana przy pomocy algorytmu \emph{RMWH} dla 1000 rund, ze zmiennym parametrem $c$ w czasie, tj. co 100 iteracji parametr $c$ zmniejszany jest o $75\%$. Pozwala to na zastrzeżenie wyników oraz poprawę parametrów, których duży skok prowadziłby do znaczącego spadku funkcji wiarygodności. Wynikowa kalibracja zmiennych (najlepszy wynik uzyskany w przebiegu) sugeruje spadek znaczenia zmiennych $\rho_g$ oraz $\rho_{\pi}$ odpowiadających za informację zwrotną z luki PKB oraz inflacji, dodatkowo możemy zauważyć znaczny wzrost wartości w pierwszych iteracjach. Parametr $\omega$ opisujący w modelu ocenę spadku użyteczności pracy był jednym z najbardziej niestabilnych, jego wartość pomiędzy rundami algorytmu znacznie się różniła. Parametr $\rho_A$ opisujący zwrot produktywności z poprzedniego przedziału zachowywał stałą wartość po wstępnym spadku.


Jednym z problemów napotkanych przy przygotowaniu estymacji było wskazanie wejściowej macierzy kowariancji $\hat{\Sigma}$ dla algorytmu \emph{Random Walk MH}. Zastosowane parametry powstały na bazie rozkładu a priori wskazanego w pracy. Drugim z problemów był mały wpływ niektórych zmiennych na wartość funkcji wiarygodności, co połączone z dużą wartością macierzy kowariancji $\hat{\Sigma}$ prowadziło do dużej zmienności w wartości zaakceptowanej próbki. Ten efekt był najbardziej zauważalny dla wartości zmiennej $\omega$, która zmieniała się drastycznie w czasie działania estymacji nawet na etapie wygaszania aproksymacji, kiedy zmienna $c$ była już znacząco obniżona zgodnie ze zastosowaną wersją opisaną w \ref{sec:rwmh_algorithm}.

Kolejnym z problemów był brak możliwości interpretacji wartości parametrów, które opisują własności agentów oraz komponentów modelu i nie są wartościami obserwowalnymi gospodarki. Dodatkowo model opierał się o rozbudowaną teorię ekonomiczną i realne dane gospodarcze, co znacznie utrudniało zrozumienie otrzymanych wyników. 

Ostatnim z napotkanych problemów było wskazanie macierzy kowariancji szoku. W przypadku wykonania estymacji przy użyciu programu \emph{Dynare}, na wyjściu analizy modelu wskazana została przeskalowana macierz kowariancji, stąd pojawiło się pytanie, czy nie jest wymagany dodatkowy algorytm pozwalający na konstrukcję i poprawę szoków wprowadzanych do filtru z informacji uzyskanych w pracy. 

% Jeden z problemów napotkanych przy wykonywaniu estymacji było wskazanie wejściowej macierzy kowariancji $\Sigma$ potrzebnej w tym wariancie algorytmu Metropolisa-Hastingsa. Zastosowane parametry wejściowe doprowadziły do dużej zmiany wartości parametrów z małą wartością w etapie wygaszania aproksymacji. Model opierał się o rozbudowaną teorię ekonomiczną, co prowadziło do dosyć trudnego wyszukiwania błędów i popraw oraz wykluczało prostą interpretację. Dodatkowo budowa macierzy kowariancji szoku z dostępnych danych doprowadziła do pytania, czy nie jest wymagany dodatkowy algorytm pozwalający na konstrukcję i poprawę szoków wprowadzanych do filtru z informacji uzyskanych w pracy.
